\documentclass{article}

\usepackage{amsmath}

\begin{document}

We study several variants of Gotoh's algorithm. We give the algorithm for two
fixed sequences $a$ and $b$ with lengths $n$ and $m$ respectively.

All algorithms are given in their similarity versions. In particular, we
assume the (dis)similarity of a gap of length k is $\alpha+k\beta$, where
$\alpha,\beta<0$.

\section{Unambigous Gotoh}

The usual version of Gotoh's algorithm is not unambigous, since one alignment
can yield the optimal score for more than one matrix entry. 

The idea of the algorithm is that the entries $i,j$ of the three matrices
$M, A,$ and $B$ each represent maximal similarities of alignments of
prefixes $a_1,\dots,a_i$ with $b_1,\dots,b_j$. For $M$ the alignments end in
a match, for $A$ the alignments end in a gap in the second row, for $B$
the alignments end in a gap in the first row.

Note that $M_{0,0}$ is redefined to represent the empty alignment.

The optimal score is now the maximum of $M_{n,m}$, $A_{n,m}$, and $B_{n,m}$ and
not necessarily found in $M_{n,m}$ (anymore).

\begin{align*}
\intertext{Initialization}
M_{0,0} = 0\\
\forall i>0: M_{i,0} &= -\infty\\
\forall j>0: M_{0,j} &= -\infty\\
%
\forall j\geq 0: A_{0,j} &= -\infty\\
\forall i>0: A_{i,0} &= \alpha+i\cdot\beta\\
%
\forall i\geq 0: B_{i,0} &= -\infty\\
\forall j>0: B_{0,j} &= \alpha+j\cdot\beta\\
%
\intertext{Recursion}
  \forall i,j>0: M_{i,j} & = \max
  \begin{cases}
    M_{i-1,j-1} + \sigma(i,j) \\
    A_{i-1,j-1} + \sigma(i,j) \\
    B_{i-1,j-1} + \sigma(i,j) \\
  \end{cases}
\\
  \forall i,j>0: A_{i,j} & = \max
  \begin{cases}
    M_{i-1,j} + \alpha + \beta \\
    A_{i-1,j} + \beta \\
    B_{i-1,j} + \alpha + \beta \\
  \end{cases}
\\
  \forall i,j>0: B_{i,j} & = \max
  \begin{cases}
    M_{i,j-1} + \alpha + \beta \\
    A_{i,j-1} + \alpha + \beta \\
    B_{i,j-1} + \beta \\
  \end{cases}
\end{align*}

Note that we allow alignments
\begin{verbatim}
...A-...
...-B...
\end{verbatim}
and
\begin{verbatim}
...-A...
...B-...
\end{verbatim}
Interestingly, these alignments are not supported in earlier versions of the
partition variant of Gotoh's algorithm. ( As a side note: for computing edge
probabilities, there is no difference whether we allow both symmetrical
alignments or only one of them. However there is a difference, if we forbid
these alignments. )

\section{Partition Variant of Gotoh}

The unambigous version of Gotoh can be directly translated into its partition
variant. Let $T$ denote the temperature of the system.

\newcommand{\ef}[1]{\bar#1}
We define
\begin{displaymath}
  \ef{x} = \exp(x/T)
\end{displaymath}
%
\begin{align*}
\intertext{Initialization}
Z^M_{0,0} = 1\\
\forall i>0: Z^M_{i,0} &= 0\\
\forall j>0: Z^M_{0,j} &= 0\\
%
\forall j\geq 0: Z^A_{0,j} &= 0\\
\forall i>0: Z^A_{i,0} &= \ef{\alpha}\cdot\ef{\beta}^i\\
%
\forall i\geq 0: Z^B_{i,0} &= 0\\
\forall j>0: Z^B_{0,j} &= \ef{\alpha}\cdot\ef{\beta}^j\\
%
\intertext{Recursion}
  \forall i,j>0: Z^M_{i,j} & = \sum
  \begin{cases}
    Z^M_{i-1,j-1} \cdot \ef{\sigma(i,j)} \\
    Z^A_{i-1,j-1} \cdot \ef{\sigma(i,j)} \\
    Z^B_{i-1,j-1} \cdot \ef{\sigma(i,j)} \\
  \end{cases}
\\
  \forall i,j>0: Z^A_{i,j} & = \sum
  \begin{cases}
    Z^M_{i-1,j} \cdot \ef{\alpha} \cdot \ef{\beta} \\
    Z^A_{i-1,j} \cdot \ef{\beta} \\
    Z^B_{i-1,j} \cdot \ef{\alpha} \cdot \ef{\beta} \\
  \end{cases}
\\
  \forall i,j>0: Z^B_{i,j} & = \sum
  \begin{cases}
    Z^M_{i,j-1} \cdot \ef{\alpha} \cdot \ef{\beta} \\
    Z^A_{i,j-1} \cdot \ef{\alpha} \cdot \ef{\beta} \\
    Z^B_{i,j-1} \cdot \ef{\beta} \\
  \end{cases}
\end{align*}

The partition sum is
\begin{displaymath}
  Z = Z^M_{n,m} + Z^A_{n,m} + Z^B_{n,m}
\end{displaymath}


Its now possible to compute the probabilities for single alignment edges in
the essemble of all alignments (Boltzman-distributed according to their
similarity scores). Therefore we define matrices $Z^M_r$, $Z^A_r$, $Z^B_r$
analogously to the above matrices however on the reversed sequences.

The probability of the edge $(i,j)$ is
\begin{displaymath}
  Pr(i,j) = \frac{1}{Z}[Z^M_{i,j} \cdot ({Z^M_r}_{n-i,m-j} + {Z^A_r}_{n-i,m-j} +
  {Z^B_r}_{n-i,m-j})].
\end{displaymath}

\subsection{Local Unambiguous Gotoh}

For making Gotoh's algorithm local and maintaining unambiguity, in particular
we need to take care of treating the empty alignment and alignment including empty sequences properly.

The key is again in definining the alignments that are taken into account in
each matrix entry. 
\begin{itemize}
\item $M_{i,j}$: all (global) alignments of $a_{i'}\dots a_i$ $(i'<i)$ and
  $b_{j'}\dots b_j$ $(j'\leq j)$ that end in the match of $a_i$ and $b_j$. 
\item $A_{i,j}$: all (global) alignments of $a_{i'}\dots a_i$ $(i' \leq i)$ and
  $b_{j'}\dots b_{j}$ $(j'\leq j+1)$ that end in the alignment of $a_i$ with a gap.
\item $B_{i,j}$: all (global) alignments of $a_{i'}\dots a_{i}$
  $(i'\leq i+1)$ and $b_{j'}\dots b_j$ $(j' \leq j)$ that end in the alignment of
  a gap with $b_j$.
\end{itemize}

This definition still contains ambiguity, since for the entries $A_{i,j}$ there
is the possibility to align with the empty word $b_{j+1}\dots b_j$. This means
for a fixed $i$ the alignments with this empty word occur redundantly in all
entries $A_{i,j}$. This can be cured by subtracting $A_{i,0}$ from all $A_{i,j}
(j>0)$. (The matrix $B$ behaves symmetrically with $i$ and $j$ exchanged.)

The empty alignment is not represented in the matrix.

%
\begin{align*}
  \intertext{Initialization}
%
  M_{0,0}  &= -\infty\\
  \forall i>0: M_{i,0} &= -\infty\\
  \forall j>0: M_{0,j} &= -\infty\\
%
  \forall j\geq 0: A_{0,j} &= -\infty\\
  \forall i>0: A_{i,0} &= \alpha+\beta\\
%
  \forall i\geq 0: B_{i,0} &= -\infty\\
  \forall j>0: B_{0,j} &= \alpha+\beta\\
%
  \intertext{Recursion}
  \forall i,j>0: M_{i,j} & = \max
  \begin{cases}
    \sigma(i,j) \\
    M_{i-1,j-1} + \sigma(i,j) \\
    A_{i-1,j-1} + \sigma(i,j) \\
    B_{i-1,j-1} + \sigma(i,j) \\
  \end{cases}
\\
  \forall i,j>0: A_{i,j} & = \max
  \begin{cases}
    \alpha+\beta \\
    M_{i-1,j} + \alpha + \beta \\
    A_{i-1,j} + \beta \\
    B_{i-1,j} + \alpha + \beta \\
  \end{cases}
\\
  \forall i,j>0: B_{i,j} & = \max
  \begin{cases}
    \alpha+\beta \\
    M_{i,j-1} + \alpha + \beta \\
    A_{i,j-1} + \alpha + \beta \\
    B_{i,j-1} + \beta \\
  \end{cases}
\end{align*}

\newpage
\subsection*{Example: aligning ``1234'' and ``abc''}
{\footnotesize
\begin{verbatim}
========================================================================
 M
     -                a                   b                 c

-     
      

1                     1                   -1;1              --1 -1;1
                      a                   ab b              abc bc c

2                     12;2                12;-12 12;1-2;2
                      -a a                ab a-b -b -ab b

3                     123 23;3            123 23;123 -123 123 1-23 23;12-3 2-3;3
                      --a -a a            -ab ab a-b a--b --b -a-b -b --ab -ab b

4                     1234 234 34;4
                      ---a --a -a a

========================================================================
 A
     -                a                   b                 c

-    
     

1    1                -1 1                --1 -1 1          ---1 --1 -1 1
     -                a- -                ab- b- -          abc- bc- c- -

2    12 2             12;-12 12;1-2;2     -12 12;--12 -12 12;1-2 -1-2 1-2 1--2 -2;2
     -- -             a- a-- -- -a- -     ab- b- ab-- b-- -- ab- a-b- -b- ab-- b- -

3    123 23 3         123 23;123 -123 123 1-23 23;3
     --- -- -         -a- a- a-- a--- --- -a-- -- -

4    1234 234 34 4    1234 234 34;1234 234 1234 -1234 1234 1-234 234 34;4
     ---- --- -- -    --a- -a- a- -a-- a-- a--- a---- ---- -a--- --- -- -

========================================================================
 B
     -                a                   b                 c

-                     -                   -- -              --- -- -
                      a                   ab b              abc bc c

1                     1-                  1-;-1- 1-;1--;-
                      -a                  ab a-b -b -ab b

2                     12- 2-              12- 2-;12- -12- 12- 1-2- 2-;12-- 2--;-
                      --a -a              -ab -b a-b a--b --b -a-b -b --ab -ab b

3                     123- 23- 3-
                      ---a --a -a 

4                     1234- 234- 34- 4-
                      ----a ---a --a -a
\end{verbatim}
}

\section{Partition Variant of Local Gotoh}

Again we can give a direct translation for computing the partition function.
%
\begin{align*}
  \intertext{Initialization}
%
  Z^M_{0,0} &= 0\\
  \forall i>0: Z^M_{i,0} &= 0\\
  \forall j>0: Z^M_{0,j} &= 0\\
%
  \forall j\geq 0: Z^A_{0,j} &= 0\\
  \forall i>0: Z^A_{i,0} &=  \sum_{1\leq i'\leq i} \ef{\alpha}\cdot \ef{\beta}^{i'}\\
%
  \forall i\geq 0: Z^B_{i,0} &= 0\\
  \forall j>0: Z^B_{0,j} &= \sum_{1\leq j'\leq j}\ef{\alpha}\cdot \ef{\beta}^{j'}\\
%
  \intertext{Recursion}
  \forall i,j>0: Z^M_{i,j} & = \sum
  \begin{cases}
    \ef{\sigma(i,j)} \\
    Z^M_{i-1,j-1} \cdot \ef{\sigma(i,j)} \\
    Z^A_{i-1,j-1} \cdot \ef{\sigma(i,j)} \\
    Z^B_{i-1,j-1} \cdot \ef{\sigma(i,j)} \\
  \end{cases}
\\
  \forall i,j>0: Z^A_{i,j} & = \sum
  \begin{cases}
    \ef{\alpha}\cdot \ef{\beta} \\
    Z^M_{i-1,j} \cdot \ef{\alpha} \cdot \ef{\beta} \\
    Z^A_{i-1,j} \cdot \ef{\beta} \\
    Z^B_{i-1,j} \cdot \ef{\alpha} \cdot \ef{\beta} \\
  \end{cases}
\\
  \forall i,j>0: Z^B_{i,j} & = \sum
  \begin{cases}
    \ef{\alpha}\cdot \ef{\beta} \\
    Z^M_{i,j-1} \cdot \ef{\alpha} \cdot \ef{\beta} \\
    Z^A_{i,j-1} \cdot \ef{\alpha} \cdot \ef{\beta} \\
    Z^B_{i,j-1} \cdot \ef{\beta} \\
  \end{cases}
\end{align*}

One can finally calculate the partition function $Z$ and edge probabilities
$Pr(i,j)$ (where we again use the corresponding matrices $Z^M_r$, $Z^A_r$, and
$Z^B_r$ for the reversed
sequences) by
\begin{align*}
  Z &= 1 + \sum(Z^M_{i,j} + Z^A_{i,j} + Z^B_{i,j}) - m\sum Z^A_{i,0} - n\sum Z^B_{0,j}
\intertext{and}
  Pr(i,j) & = \frac{1}{Z}[Z^M_{i,j} \cdot ({Z^M_r}_{n-i,m-j} + {Z^A_r}_{n-i,m-j} +
  {Z^B_r}_{n-i,m-j} + 1)].
\end{align*}
Note the additions of $1$ which correct for the empty alignment!


\end{document}

%%% Local Variables: 
%%% mode: latex
%%% TeX-master: t
%%% End: 
