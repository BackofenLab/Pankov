\documentclass{article}

\usepackage{theorem}
\usepackage{amsmath}

\newtheorem{definition}{Definition}
\newtheorem{proposition}{Proposition}

\newcommand{\Begin}{\operatorname{Begin}}
\newcommand{\End}{\operatorname{End}}

\newcommand{\accbegin}{\operatorname{begin\_acc}}
\newcommand{\accend}{\operatorname{end\_acc}}
\newcommand{\expbeginacc}{\operatorname{exp\_begin\_acc}}
\newcommand{\expendacc}{\operatorname{exp\_end\_acc}}

\begin{document}

It seems that the problem of local maximum expected accuracy alignment is not
discussed thoroughly in the literature yet.

One possible approach is to build on probabilities of local alignments and
define local accuracy by introducing begin and end accuracy.

\section{Preliminaries}

\begin{definition}
  A \emph{local alignment} of sequences $x$ and $y$ is an alignment of two
  subsequences of $x$ and $y$ that begins and ends with a match (?).
\end{definition}

In the following we fix two sequences $x$ and $y$.
\begin{definition}[Notation]
  $Pr[A]$ denotes the probabilitiy of the local alignment $A$ of $x$ and $y$.
  $Pr[x_i\sim y_j]$ denotes the probability of aligning $x_i$ to $y_j$ in a
  local alignment of $x$ and $y$.
  \\
  $\Begin_\chi(A)$ is the first position of the subsequence of sequence
  $\chi\in\{x,y\}$ in a local alignment $A$, $\End_\chi(A)$ the last one.
  $\Begin_\chi$ and $\End_\chi$ denote the corresponding random variables.
\end{definition}

Probabilities of local alignments and local alignment edges can be computed
efficiently using the local partition function variant of Gotoh's algorithm.
Similarly we compute probabilities $Pr[\Begin_\xi=k]$ and $Pr[\End_\xi=l]$.

Note that we should use a version of Gotoh that accounts to the above
definition of local alignment (begin and end with match!). Currently, we have a
version that defines local alignment just as an alignment of subsequences.

\section{Begin and End Accuracy}
\label{sec:beg-end-acc}

\begin{definition}
  \begin{align*}
    \accbegin_\chi(A,A') &= 
    \begin{cases}
      -1 & \text{if $Begin_\xi(A)\neq \Begin_\xi(A')$}\\
      0 & \text{otherwise.}
    \end{cases}
    \\
    \expbeginacc_\chi(A) & = \sum_{A'} \accbegin_\chi(A,A') Pr[A']\\
  \end{align*}
\end{definition}

\begin{proposition}
  \begin{align*}
    \expbeginacc_\chi(A)
    & = \sum_{A': Begin_\chi(A)\neq \Begin_\chi(A')} -Pr[A']\\
    & = \sum_{k\neq\Begin_\chi(A)} \sum_{\Begin_\chi(A')=k} -Pr[A']\\
    & = \sum_{k\neq\Begin_\chi(A)} - Pr[\Begin_\chi=k]
  \end{align*}
\end{proposition}

For the recursion, we add the position wise expected begin accuracy
\begin{proposition}
  \begin{align*}
    D_{ij} = \max 
    \begin{cases}
      Pr[x_i\sim y_j] & \text{begin with match of i and j}\\
      D_{i-1 j-1} + Pr[x_i\sim y_j] - Pr[\Begin_x=i] - Pr[\Begin_y=j] &\\
      D_{i-1 j} - Pr[\Begin_x=i] &\\
      D_{i j-1} - Pr[\Begin_y=j] &\\
    \end{cases}
  \end{align*}
\end{proposition}



\subsection{Improved version}

Begin and end accuracy can be defined in a more general way that still allows
efficient treatment.
\begin{definition}
  \begin{align*}
    \accbegin_\chi(A,A') &= f(\Begin(A)_\chi,\Begin_\chi(A'))
  \end{align*}
\end{definition}

Then,
\begin{align*}
    \expbeginacc_\chi(A) & = \sum_{A'} \accbegin_\chi(A,A') Pr[A']\\
    & = \sum_{k} \sum_{\Begin_\chi(A')=k} Pr[A'] \\
    & = \sum_{k} f(\Begin_\chi(A),k) \cdot ( \sum_{\Begin_\chi(A')=k}
    Pr[A'] )\\
    & = \sum_{k} f(\Begin_\chi(A),k) \cdot Pr[\Begin_\chi=k].
\end{align*}

In the recursion equation the term, which can be precomputed, is added at the
beginning of the local alignment.
\begin{proposition}
  \begin{align*}
    D_{ij} = \max 
    \begin{cases}
      Pr[x_i\sim y_j] + \sum_{k} f(i,k) \cdot Pr[\Begin_\chi=k] & \text{begin with match of i and j}\\
      D_{i-1 j-1} + Pr[x_i\sim y_j]&\\
      D_{i-1 j} &\\
      D_{i j-1} &\\
    \end{cases}
  \end{align*}
\end{proposition}



\end{document}

%%% Local Variables: 
%%% mode: latex
%%% TeX-master: t
%%% End: 
