%%This is a very basic article template.
%%There is just one section and two subsections.
\documentclass{article}

\usepackage{amsmath,amsthm}


\begin{document}
\title{Normalized Sequence-Structure Alignment for RNAs}
\author{Steffen Heyne and Sebastian Will}
\maketitle

\section{Preliminaries}

The idea for normalized local alignment (NLA) for RNAs is based on ideas of
Arslan et \emph{al.} This approach make use of the Dinkelbach
algorithm and fractional programming to get optimal normalized scores.

\subsubsection*{Sequences}
	
Given pair of RNA sequences $A$ and
$B$ of length $n=\vert A \vert$ and $m=\vert B \vert$. Subsequences are written
as $a_i\dots a_j$ and $b_k\dots b_l$.

\subsubsection*{Alignment Graph}

An alignment graph $\mathcal{G}_{A,B}$ is a directed acyclic graph  having
$(n+1)(m+1)$ lattice points $(u,v)$ for $0\leq u \leq n$ and $0\leq v \leq m$
as vertices. Possible arcs in $\mathcal{G}_{A,B}$ are horizontal arcs, vertical
arcs and diagonal arcs\ldots

\subsubsection*{Alignment vector}

We say that $(S,x,y,z)$ is an alignment vector for two subsequences $a_i\dots a_j$
and $b_k\dots b_l$, if there exists an alignment path in
the alignment graph $\mathcal{G}_{A,B}$ between the vertices $(i-1,k-1)$ and
$(j,l)$ with $x$ matches that have a similarity of $S$ by using $y$ indels and
$z$ gap openings.

\subsubsection*{Alignment}

The trace of an alignment vector $(S,x,y,z)$ in the alignment graph
$\mathcal{G}_{A,B}$ is denoted by $\mathcal{A}_{i,j,k,l}$ and is called 
alignment for the sequences $a_i\dots a_j$ and $b_k\dots b_l$. If $j-i+1 < n$
or $l-k+1 < m$ the alignment $\mathcal{A}_{i,j,k,l}$ is called \emph{local}
alignment in respect to $A$ and $B$. The alignment length is denoted by
$\vert \mathcal{A}_{i,j,k,l}\vert$. The $i$th column in $\mathcal{A}$ is
denoted by $(p,q)_i$ with $i\leq \vert \mathcal{A}\vert$.

\subsubsection*{Scoring}

In LocARNA, only similarity scores are used instead of match and mismatch
scores. The similarity of two nucleotides $p=a_i$ and $q=b_j$ is
denoted by $\sigma_{p,q}$.

Indel penalty: $\gamma$

Gap opening penalty: $\alpha$

\subsubsection*{Similarity}

The similarity $S$ of an alignment (path) is defined as follows. For each
diagonal arc in the alignment path in $\mathcal{G}_{A,B}$ between the
vertices $(u-1,v-1)$ and $(u,v)$ we have a match operation $\sigma_{p,q}$ with
$p=a_u$ and $q=b_v$. With $f_{p,q}$ we indicate the frequency of the same match
operation in the alignment (path) between the nucleotides $p$ and $q$.

An alignment (path) has a similarity of

\begin{equation}
S = \sum\limits_{(p,q) \in \mathcal{A}}\sigma_{p,q}f_{p,q} 
\end{equation}

This is equal to sum over all similarity scores in match columns 

\begin{equation}
S = \sum\limits_{\substack{i\leq|\mathcal{A}|\\ (p,q)_i \in
\mathcal{A}}}\sigma_{p,q}
\end{equation}

and the number of match columns (non gap) is equal to

\begin{equation}
x = \sum\limits_{(p,q) \in \mathcal{A}}f_{p,q} 
\end{equation}

\subsubsection*{Alignment Score}

An alignment vector $(x,y,z)$ has a score defined by $\gamma$ and $\alpha$,
which is defined as:

\begin{equation}
\textsl{SCORE}(x,y,z) = \sum_x \sigma_{p,q}(x) - \gamma \cdot y-
\alpha\cdot z
\end{equation}

The alignment $\mathcal{A}$ contains $y$ indels over $z$ individual stretches
of gaps.
		
The maximum score between subsequences $a_i \dots \a_j$ and $b_k \dots b_l$ is
the maximum score of all alignment vectors for these two subsequences:

\begin{equation} 
S_{\sigma,\gamma,\alpha}(a_i \dots a_j, b_k \dots b_l) = \max \lbrace 
\textsl{SCORE}(x,y,z) \vert (x,y,z) \text{ is alignment vector}\rbrace
\end{equation}

\subsubsection*{Length function}

We define a length function with respect to some positive contant $L$ as

\begin{equation}
\textsl{LENGTH}_{L}(a_i \dots a_j,b_k \dots b_l)= (j-i+1)+(l-k+1)+L
\end{equation}

\subsubsection*{Local Alignment}

Local Alignment (LA) problem is to find two subsequences with the highest
similarity score LA* defined as:
\begin{equation}
\text{LA}^{*}_{\sigma,\gamma, \alpha}(a,b) = \max \lbrace
\text{SCORE}(x,y,z)\ \vert \ (x,y,z) \text{ is align. vector }\rbrace
\end{equation}

\subsubsection*{Normalized Score}
 
 A normalized score (with repect to L) $\text{NS}_L$ of two subsequences
 $a_i\dots a_j, b_k \dots b_l$ is the ratio of their maximum score to the value
 of $\text{LENGTH}_L$ for these subsequences 


\begin{equation}
\textsl{NS}_{\sigma,\gamma,\alpha,L}(a_i \dots a_j,b_k \dots b_l)  =
\frac{S_{\sigma,\gamma,\alpha}(a_i \dots a_j, b_k \dots b_l)}
{\textsl{LENGTH}_{L}(a_i \dots a_j,b_k \dots b_l)}
\end{equation}

\subsubsection*{Normalized Local Alignment}

Normalized Local Alignment (NLA) is to find two subsequences $a_i \dots a_j$
and $b_k \dots b_l$ which have the maximal normalized score for all possible
subsequences:

\begin{equation}
\text{NLA}^{*}_{\sigma,\gamma,\alpha,L}(a,b) = \max \lbrace
\textsl{NS}_{\sigma,\gamma,\alpha,L}(a_i \dots a_j,b_k \dots b_l) \rbrace
\end{equation}

\subsubsection*{Observations}

if $(x,y,z)$ is an alignment vector, then

\begin{equation}
(j-i+1) + (l-k+1)= 2 x + y 
\end{equation}

and 
\begin{equation}
\textsl{LENGTH}_{L}(a_i \dots a_j,b_k \dots b_l) =
2x + y + L
\end{equation}

\subsection{Algorithm}

LA and NLA are optimzation problems. 

\begin{equation}
\text{LA}^{*}_{\sigma, \gamma, \alpha}(a,b): \text{maximize}\ \sum_x
\sigma_{p,q}(x) -\gamma y - \alpha z
\end{equation}

\begin{equation}
\text{NLA}_{\sigma,\gamma,\alpha,L}(a,b): \text{maximize}\ \frac{\sum_x
\sigma_{p,q}(x)  - \gamma \cdot y- \alpha \cdot z}{2 x + y + L}
\end{equation}

For a given $\lambda$ we define the \emph{parametric local alignment} problem

\begin{equation}
\text{LA}_{\sigma,\gamma,\alpha,L}(\lambda)(a,b): \text{maximize}\ \sum_x
\sigma_{p,q}(x) - \gamma\cdot y- \alpha\cdot z - \lambda\left( 2
x +y + L \right)
\end{equation}

\begin{equation}
\text{LA*}(\lambda) = max\lbrace\sum_x \sigma_{p,q}(x) -
\gamma\cdot y- \alpha\cdot z - 2 \lambda x -\lambda y - \lambda
L \rbrace
\end{equation}



\begin{equation}
 = max \left\lbrace \left( \sum_x \sigma_{p,q}(x) -2 \lambda\right) x -(
 \gamma +\lambda) y \right\rbrace - \alpha z  - \lambda L
\end{equation}


\begin{equation}
\text{LA}^{*}_{\sigma',\gamma',\sigma'} (a,b) - \alpha z  - \lambda L
\end{equation}

\begin{equation}
\text{where } \sigma'_{x,y} = \sigma_{x,y} - 2\lambda  \ \ \ \ \  \gamma' =
\gamma -
\lambda
\end{equation}



\end{document}
